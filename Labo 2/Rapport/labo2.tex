\documentclass[french,10pt,a4paper]{article}
\usepackage{amsmath}
\usepackage[T1]{fontenc}
\usepackage{lmodern}
\usepackage[margin=3cm]{geometry}
\usepackage{graphicx}
\usepackage{hyperref}
\usepackage{amssymb}
\usepackage{xcolor}
\usepackage{babel}
\usepackage[font=small,labelfont=bf]{caption}

\title{\centering\includegraphics[width=5cm]{images/hepl_logo.png}\\Transistors à effet de champ: Labo 2}
\author{AL-ZUBAIDI Ahmed, DUSPEAUX Antoine, RONDIA Arthur}
\begin{document}
	\maketitle
	\section{Partie A: Etude DC}
		\subsection{Simulations Multisim}
			\subsubsection{Simulation DC-Sweep}
			\subsubsection{Evaluation des courbes et zones}
			\subsubsection{Mesures à partir de la courbe}
		\subsection{Manipulations physiques}
			\subsubsection{Tableau des caractéristiques DC}
			\subsubsection{Circuit sur breadboard}
			\subsubsection{Courbe $V_\text{ds}$/$V_\text{Id}$}
			\subsubsection{Evaluation pour $V_\text{GS}$ = 4, 4.5 et 5 V}
			\subsubsection{Courbes $V_\text{ds}$/$V_\text{Id}$ pour différents $V_\text{gs}$}
			\subsubsection{Evaluation des $R_\text{ds}$ et $I_\text{ds}$ des courbes et zones}
			\subsubsection{Courbe de transfert $V_\text{gs}$/$V_\text{Id}$ avec $V_\text{ds}$ = 10 V}
			\subsubsection{Evaluation de $V_\text{th}$ et $g_\text{m}$}
		
	
	\section{Partie B: Etude AC}
		\subsection{Simulations Multisim}
			\subsubsection{Simulation transitoire de $V_\text{G}$, $V_\text{D}$ et $V_\text{1}$}
			\subsubsection{Description des imperfections dues aux capacités parasites}
			\subsubsection{Constats pour $R_\text{2}$ = 100}
			\subsubsection{Constats pour $R_\text{2}$ = 100k, $R_\text{1}$ = 10}
			\subsubsection{Estimation du courant maximum de charge de $C_{\text{gs}}$ et $C_{\text{gd}}$}
			\subsubsection{Simulation du courant $I_\text{g}$ et comparaison au développement théorique} 
		\subsection{Manipulations physiques}
			\subsubsection{Circuit sur breadboard}
			\subsubsection{Mesure à l'oscilloscope des tensions $V_\text{G}$, $V_\text{1}$ et $V_\text{D}$,$V_\text{1}$ et comparaison aux simulations}
			\subsubsection{Mesure à l'oscilloscope de $V_\text{G}$ et $V_\text{D}$ pour démontrer les effets capacitifs parasites}

		
	\section{Conclusion}
		
\end{document}