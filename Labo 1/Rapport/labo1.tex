\documentclass[french,10pt,a4paper]{article}
\usepackage{amsmath}
\usepackage[T1]{fontenc}
\usepackage{lmodern}
\usepackage[margin=3cm]{geometry}
\usepackage{graphicx}
\usepackage{hyperref}
\usepackage{amssymb}
\usepackage{xcolor}
\usepackage{babel}
\usepackage[font=small,labelfont=bf]{caption}

\title{\centering\includegraphics[width=5cm]{images/hepl_logo.png}\\Transistors à effet de champ: Labo 1}
\author{AL-ZUBAIDI Ahmed, DUSPEAUX Antoine, RONDIA Arthur}
\begin{document}
	\maketitle
	\section{Partie A: Etude DC}
		\subsection{Simulations Multisim}
			\subsubsection{Justification des choix pour le montage diviseur en tension}
			\subsubsection{Droite de charge sur $V_\text{ds}$ et identification de $Q$}
			\subsubsection{Simulation du montage sur Multisim}
		\subsection{Manipulations physiques}
			\subsubsection{Circuit sur breadboard}
			\subsubsection{Mesure des paramètres du circuit et justification des différences}
			\subsubsection{Calcul de la puissance dissipée}
		
	
	\section{Partie B: Etude AC}
			\subsubsection{Montage amplificateur source commune, schéma équivalent et impédances}
			\subsubsection{Estimation de la transconductance et calcul du gain théorique}
			\subsubsection{Mesures à l'oscilloscope pour f = 20 kHz}
			\subsubsection{Mesures à l'oscilloscope pour un balayage entre 100 Hz et 200 kHz, diagrammes de Bode et justifications}

		
	\section{Conclusion}
		
\end{document}